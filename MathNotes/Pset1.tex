\documentclass[12pt]{article}
\author{\normalsize You know who}
\title{\huge Problem set from Y11 3U finals}
\date{\today}

\usepackage{amsmath, amssymb, graphicx}
\graphicspath{assets/}

\usepackage[a4paper, margin=0.8in]{geometry}
\usepackage[skip=0pt]{parskip}
\usepackage{setspace}
\setstretch{1.25}
\raggedright

\begin{document}
\maketitle
\section*{List of problems}
\large

\begin{enumerate}
	\item For $P(x) = (4x + \frac{1}{k})^n$, the following is true: \\
	      The coefficient of $x^4$ is 8 times that of $x^3$ and the coefficient of $x^2$ is 24 times that of $x$. \\
	      Find $n$ and $k$. \begin{flushright} (3 Marks) \end{flushright}

	\item For $0\leq x<\pi$, evaluate $x$ so that $\sin{4x} = \sin{2x}$. \begin{flushright} (2 Marks) \end{flushright}
\end{enumerate}

% PAGE 1 OF SOLN 1
\newpage
\section*{Soln. for problem 1}
\normalsize

Understand that $P(x)$ may be expanded as:
\begin{align*}
	\displaystyle\sum_{i=0}^{n}{\binom{n}{i} (4x)^{n-i} \times k^{-i}}
\end{align*}
And that $\binom{n}{i} = \binom{n}{n-i}$ by the symmetric distribution of Pascal's triangle. \\
Thus taking the binomial coefficients of $x^4, x^3, x^2, x$ the following is obtained:
\begin{itemize}
	\item For $x^4$:
	      \begin{equation}
		      \begin{split}
			      C & = \binom{n}{4} \times 4^4 \times k^{4-n}                \\
			        & = \frac{n!}{(4!)(n-4)!} \times 256 \times k^{4-n}       \\
			        & = \frac{n(n-1)(n-2)(n-3)}{4!} \times 256 \times k^{4-n} \\
			        & = \frac{32(n)(n-1)(n-2)(n-3)}{3} \times k^{4-n}
		      \end{split}
		      \label{eq:No.1}
	      \end{equation}

	\item For $x^3$:
	      \begin{equation}
		      \begin{split}
			      C & = \binom{n}{3} \times 4^3 \times k^{3-n}            \\
			        & = \frac{n!}{(3!)(n-3)!} \times 64 \times k^{3-n}    \\
			        & = \frac{(n)(n-1)(n-2)}{3!} \times 64 \times k^{3-n} \\
			        & = \frac{32(n)(n-1)(n-2)}{3} \times k^{3-n}
		      \end{split}
		      \label{eq:No.2}
	      \end{equation}

	\item For $x^2$:
	      \begin{equation}
		      \begin{split}
			      C & = \binom{n}{2} \times 4^2 \times k^{2-n}         \\
			        & = \frac{n!}{(2!)(n-2)!} \times 16 \times k^{2-n} \\
			        & = \frac{(n)(n-1)}{2!} \times 16 \times k^{2-n}   \\
			        & = 8(n)(n-1) \times k^{2-n}
		      \end{split}
		      \label{eq:No.3}
	      \end{equation}

	\item For $x$:
	      \begin{equation}
		      \begin{split}
			      C & = \binom{n}{1} \times 4^1 \times k^{1-n} \\
			        & =  4nk^{1-n}
		      \end{split}
		      \label{eq:No.4}
	      \end{equation}
\end{itemize}

% PAGE 2 OF SOLN 1
\newpage
And now consider that:
\begin{itemize}
	\item \textbf{Equation \ref{eq:No.1} = 8 $\times$ Equation \ref{eq:No.2}}
	\item \textbf{Equation \ref{eq:No.3} = 24 $\times$ Equation \ref{eq:No.4}}
\end{itemize}
Now forming two new equations:
\begin{itemize}
	\item Equation 5
	      \begin{equation}
		      \begin{split}
			       & \frac{32(n)(n-1)(n-2)(n-3)}{3} \times k^{4-n} = \frac{256(n)(n-1)(n-2)}{3} \times k^{3-n} \\ \\
			       & (n-3)k^{4-n} = 8k^{3-n}                                                                   \\ \\
			       & \frac{n-3}{8} = k^{-1}                                                                    \\ \\
			       & \frac{8}{n-3} = k
		      \end{split}
	      \end{equation}

	\item Equation 6
	      \begin{equation}
		      \begin{split}
			       & 8(n)(n-1) \times k^{2-n} =  96nk^{1-n} \\ \\
			       & (n-1)k^{2-n} = 12k^{1-n}               \\ \\
			       & \frac{n-1}{12} = k^{-1}                \\ \\
			       & \frac{12}{n-1} = k
		      \end{split}
	      \end{equation}

	\item And thus:
	      \begin{align*}
            & \frac{12}{n-1} = \frac{8}{n-3} = k \\ \\
            & \frac{3}{2} = \frac{n-1}{n-3} \Rightarrow 2(n-1) = 3(n-3) \Rightarrow 2n - 2 = 3n - 9 \\ \\ 
            & \therefore n = 7 \\ \\
            & \therefore k = 2
	      \end{align*}
\end{itemize}

% SOLN 2
\newpage
\section*{Soln. for problem 2}

Considering the domain $0 \leq x<\pi$ as $0 \leq 2x<2\pi$, \\
First rewrite $\sin{4x}$ as $2\sin{(2x)}\cos{(2x)}$, and thus:
\begin{itemize}
	\item First define LHS and RHS:
	      \begin{align*}
		      LHS & = \sin{4x}=2\sin{(2x)}\cos{(2x)} \\
		      RHS & =\sin{2x}
	      \end{align*}

	\item Thus the following cases are trivially obtained:
	      \begin{align*}
		       & \sin{2x} = 0           \\
		       & \cos{2x} = \frac{1}{2}
	      \end{align*}

	\item $\therefore$ For each of the following cases:
	      \begin{enumerate}
		      \item $\displaystyle\sin{2x} = 0$
		            \begin{gather*}
			            \sin^{-1}{0} = 2x \\ \\
			            \therefore 2x = 0, \pi \\ \\
			            x = 0, \frac{\pi}{2}
		            \end{gather*}
		      \item $\displaystyle\cos{2x} = \displaystyle\frac{1}{2}$
		            \begin{gather*}
			            \cos^{-1}{\frac{1}{2}} = 2x \\ \\
			            \therefore 2x = \frac{\pi}{3}, \frac{5\pi}{3} \\ \\
			            x  = \frac{\pi}{6}, \frac{5\pi}{6}
		            \end{gather*}
	      \end{enumerate}

	\item Thus:
	      \begin{gather*}
		      x = 0, \frac{\pi}{6}, \frac{\pi}{2}, \frac{5\pi}{6}
	      \end{gather*}

\end{itemize}

\end{document}
